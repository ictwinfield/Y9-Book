\chapter{Algebra 1}
\section{Notation}
It is importatant that when we see some algebra on the page or board that we all agree what it represents.  In this section we will introduce simply algebraic notation.
\subsection{A letter can have a value}
Imagine that apples come in bags of five that are all the same weight and that on the bag it gives their individual weight.  How would we calculate the total weight of the apples in a bag.  We would multiply the weight by 5. 

We have just described in words how to get the total weight, algebra will allow us to do this in a simplier way.  We can do this by letting 'a' represent the weight of the apple, so now the we get the equation:
$$\mbox{Total Weight} = a \times 5$$
Better still we can let 'T' represent the total weight. We now get:
$$T=a \times 5$$
\subsection{Four rules}

Addition is just what you are used to and is shown with a plus sign (+)

\noindent Subtraction is just what you are used to and is shown with a minus sign (-)
\noindent Multiplication is slightly different.  In algebra we don't use a multiplication sign we just leave it out.  It is understood that if two values are next to each other then we need to multiply them together.

For example:

$5a$ means 5 $\times$ the value of a

$ab$ means the value of a $\times$ the value of b

\noindent Something else to remember with multiplication is that we would usually put the letters in alphabetical order and we would also put numbers first.

So:
$a5$ becomes $5a$

$yx$ becomes $xy$

$x5a$ becomes $5ax$

\noindent So our equation from before becomes $$T = 5a$$

\noindent Division is simply shown as a fraction.  So $\frac{a}{b}$ means $a \div b$
\subsection{Exercise}
Write each of following statements using algebra
\begin{multicols}{2}
\begin{enumerate}
  \item a plus 6
  \item 5 minus b
  \item x multiplied z
  \item t divided by 4
  \item the total of a and b
  \item the difference between 9 and p
  \item g $\times$ 7
  \item \textbf{A equals b multiplied h}
  \item \textbf{I equals V $\div$ R}
  \item \textbf{P equals the total of a, b and c}
  \item t more than q
  \item p less than 8
  \item 4 more than ab
  \item f shared by r
  \item f less than g
  \item g less than f
  \item h less than pq
  \item p times d
  \item \textbf{v equals the total of u and 'at'}
  \item \textbf{A equals bh divided by 2}
\end{enumerate}
\end{multicols}
\section{Substitution}
It this point we understand basic algebraic notation and we also know how to produce basic expressions and equations.  In this section we are going to work out the value of various expressions.  We will be given the value of the letters and we will use these.
\begin{exmp}
What is the value $ab$ if $a=3$ and $b=7$?

$ab$ means $a \times b$

So $ab = 3 \times 7$

So $ab = 21$
\end{exmp}
\begin{exmp}
Given that $a= 4$, $b = 6$ and $c= 5$, what is $P$ if $P=a+b+c$?

$P=4+6+5$

So $P=15$
\end{exmp}
\subsection{Exercise}
Find the value of the following expressions given that $a=3,b=6,c=5,d=10,h=4$
\begin{multicols}{2}
\begin{enumerate}
  \item $a+b$
  \item $5b$
  \item $a + b + c + d$
  \item $bh$
  \item $\displaystyle \frac{d}{c}$
  \item $d-h$
  \item $\displaystyle \frac{15}{a}$
  \item $12 - c$
  \item $\displaystyle \frac{d}{2}$
  \item $7+h$
\end{enumerate}
\end{multicols}
Find the value of $x$ using the following equations.
\begin{multicols}{2}
\begin{enumerate}
  \setcounter{enumi}{10}
  \item $x=7a$
  \item $x=8+c$
  \item $x=b+d$
  \item $x=abc$
  \item $\displaystyle x=\frac{18}{a}$
  \item $d+h$
\end{enumerate}
\end{multicols}
\section{Codes}
In this section we are going read a code that has been given as a series of expressions.  We will calculate the value of the expressions and then use the talbe to find out to what letter we are refering.  For example if the value of an expression is '13' we can see from the table that this must represent a 'C'.


\begin{tabular}{|c|c|c|c|c|c|c|c|c|c|c|c|c|}
  \hline
  A & B & C & D & E & F & G & H & I & J & K & L & M \\
  \hline
  7 & 5 & 13 & 35 & 1 & 42 & 63 & 3 & 9 & 2 & 12 & 30 & 27\\
  \hline
  \hline
  N & O & P & Q & R & S & T & U & V & W & X & Y & Z\\
  \hline
  0 & 4 & 33 & 11 & 6 & 8 & 10 & 21 & 25 & 13 & 22 & 18 & 28\\
  \hline
\end{tabular}

\begin{exmp}
Given that $a= 3$, $b = 15$, $c= 1$ and $d = 21$ change each line of the table into a word to decode the message.

\vspace{3 mm}

\renewcommand{\arraystretch}{3}
\begin{tabular}{|c|c|c|c|}
\hline
$a + 2$ & $b - 6$ & $3d$ &  \\
\hline
$\displaystyle\frac{b}{a}$ & $\displaystyle\frac{d}{a}$ & $20 - b$ & $a + b$ \\
\hline
\end{tabular}

\vspace{3 mm}

Let's substitute in the values

\vspace{3 mm}

\begin{tabular}{|c|c|c|c|}
\hline
$3 + 2 = 5$ & $15 - 6 = 9$ & $3 \times 21 = 63$ &  \\
\hline
$\displaystyle \frac{15}{3} = 5$ & $\displaystyle \frac{21}{3} = 7$ & $20 - 15 = 5$ &$3 + 15 = 18$ \\
\hline
\end{tabular}

\vspace{3 mm}

Change for the letters

\vspace{3 mm}

\renewcommand{\arraystretch}{1}
\begin{tabular}{|c|c|c|c|}
\hline
B & I & G &  \\
\hline
B & A & B & Y \\
\hline
\end{tabular}

\vspace{3 mm}

So the message is 'Big baby'.
\end{exmp}
\pagebreak
\subsection{Exercise}
\begin{tabular}{|c|c|c|c|c|c|c|c|c|c|c|c|c|}
  \hline
  A & B & C & D & E & F & G & H & I & J & K & L & M \\
  \hline
  7 & 5 & 13 & 35 & 1 & 42 & 63 & 3 & 9 & 2 & 12 & 30 & 27\\
  \hline
  \hline
  N & O & P & Q & R & S & T & U & V & W & X & Y & Z\\
  \hline
  0 & 4 & 33 & 11 & 6 & 8 & 10 & 21 & 25 & 13 & 22 & 18 & 28\\
  \hline
\end{tabular}

\vspace{3 mm}
\noindent Given that 
\begin{multicols}{3}
	\begin{itemize}
		\item $p = 50$
		\item $q = 27$
		\item $r = 9$
		\item $s = 15$
		\item $t = 3$
		\item $u = 2$
		\item $v = 10$
		\item $w = 24$
		\item $x = 7$
	\end{itemize}
\end{multicols}
\noindent change each line of the table into a word to find the name of this cartoon character.

\vspace{3 mm}

\renewcommand{\arraystretch}{3}
\begin{tabular}{|c|c|c|c|c|}
\hline
$\displaystyle \frac{s}{t}$ & $v - r$ & $x + t$ & $q - x - v$ & $2r$ \\
\hline
$\displaystyle \frac{p}{v}$ & $2u$ & $\displaystyle \frac{w}{6}$ & $p - q + v$& \\
\hline
\end{tabular}

\vspace{3 mm}

