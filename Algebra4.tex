\chapter{Algebra 4}
\section{Exponents}
Just like multiplication is a short hand for repeated addition, exponnents are a short hand for repeated multiplication.  The exponent tells us how many times the value has been multiplied by itself and only applies to the value immediately to its left.

\bigskip

So $5^3$ means $5 \times 5 \times 5$ 

\bigskip

and $4x^2$ means $4 \times x \times x$

\subsection{Exercise}
Exand the following expressions and if possible find the value:
\begin{enumerate}
	\item $4^3$
	\item $2^5$
	\item $x^3$
	\item $2^3 + 3^2$
	\item $y^4$
	\item $5 \times 3^2$
\end{enumerate}

\subsection{Exercise}
Write the following expressions with exponents:
\begin{enumerate}
	\item $4 \times 4 \times 4 \times 4 \times 4 \times 4$
	\item $x \times x \times x \times x$
	\item $5 \times y \times y$
	\item $a \times b \times b \times b$
	\item $2 \times 5 \times t \times t \times m$
	\item $a \times b \times a \times a \times b$
\end{enumerate}

\section{Like Terms}
The number at the front of a term is called a coefficient and the rest of the term tells us its type.  The coefficient tells us how many of a particular type we have.  For two terms to be the same the parts after the coefficient must be identical.

\begin{exmp}
$3x+4x$, since both are of type $x$, they can be joined to become $7x$
\end{exmp}
\begin{exmp}
$4x + 7y$, one is of type $x$ and one is of type $y$ so they can't be joined so it remains as $4x+7y$
\end{exmp}
\begin{exmp}
$5x^2 - 3x + 4x$, we clearly have two $x$ terms so this can be simplified to $5x^2+x$.  Now these are two different types, one $x^2$ and the other $x$, so the expressions stays as $5x^2+x$
\end{exmp}
\begin{exmp}
$8x^2y - 6xy^2$, one of these is of type $x^2y$ and the other is of type $xy^2$.  They are different so the expression does not change.
\end{exmp}
\begin{exmp}
$xy + yx$, the order is different, but these are both of type $xy$, so can be simplified to $2xy$
\end{exmp}

\subsection{Exercise}
Where possible simplify the expressions below:
\begin{enumerate}
	\item $5x+7x$
	\item $6x+4y + 3x$
	\item $4xy - 2x + y$
	\item $5x^2-3x^2+6xy+2xy$
	\item $4pq-3qp$
	\item $5x^2+4x^2+7$
	\item $6st+3s-4st+5t$
	\item $4x+5x$
	\item $6h+3h+7h$
	\item $9t-t$
	\item $5+6g-2g+7$
	\item $4-5k-3k$
	\item $3x+5x+6y-2y$
	\item $7x+3y-2x-3y$
	\item $4x^2+5x-x^2+7$
	\item $3+5xy+2x-3y+5x+8$
	\item $6x^2+3x-5x+8$
	\item $4x^2+8xy-3xy+2y^2$
	\item $5x^3-4xy+y^2-3x^2+4y^2$
	\item $4a^2b+3ab^2$
	\item $6f^2+f5-4f-5f^2$
\end{enumerate}
\section{Dimensional Analysis}
\section{Solving Linear Equations}
We have already learnt how to solve questions of the form $ax+b=c$, we are now going to learn how to solve all linear equations.

\bigskip

Let's consider equations of the form $ax+b=cx+d$.  Often in mathematics when we are trying to solve a complex problem we just try and change it to a simplier problem that we can solve, that is what we are going to do here.  So we are going to change our problem into the form $ax+b=c$, we will do this by subtracting the value of the smallest $x$ term from both sides of the equation.

\begin{exmp}
Solve the equation $5x-7=3x+13$

\bigskip

The smallest $x$ term is $3x$ so we will subtract $3x$ from both sides of the equation.

\bigskip

So we now have $2x-7=13$

\bigskip

Now $+7$ to both sides.

\bigskip

We now have $2x=20$

\bigskip

So $x=10$
\end{exmp}

\begin{exmp}
Solve the equation $2x+5=17-4x$

\bigskip

The smallest $x$ term here is not $2x$, but $-4x$ since it is negative.  So we must subtract $-4x$ from both sides so that means we must $+4x$ to both sides of the equation.

\bigskip

So we now have $6x+5=17$

\bigskip

Now we $-5$ from both sides of the equation.

\bigskip

So we have $6x=12$

\bigskip

So $x=2$
\end{exmp}

\subsection{Exercise}
Find the value of $x$ in each of the equations.
\begin{multicols}{2}
\begin{enumerate}
	\item $5x-8=x+20$
	\item $6x+4=3x+13$
	\item $3x-5=27-5x$
	\item $5x+6=3x+10$
	\item $7x-5=4x+20$
	\item $4x+6=3x+20$
	\item $5x-8=x+4$
	\item $3x+6=8x-4$
	\item $2x-6=5x-24$
	\item $3x-9=10x-58$
	\item $2x-8=4-2x$
	\item $2x-5=7-x$
	\item $6-3x=10-5x$
	\item $5x+2=x+14$
	\item $2-3x=14-7x$
	\item $8x-5=4x+7$
	\item $8-2x=2x-4$
	\item $4x+7=9x-3$
\end{enumerate}
\end{multicols}
\subsection{Brackets}
When we have equations that contain brackets it is often best, but not always, to expand the brackets first, this will usually leave us with a problem like the previous ones.  A more complicated situation might require us to group like terms.  We will look at two examples one where the bracket gets multiplied by a positive value and one where we multiply by a negative value.

\begin{exmp}
Solve the equation $3(x+4)+5=2(2x+3)-5$

\bigskip

First expand the brackets to get $3x+12+5=4x+6-5$

\bigskip

Now simplify both sides of to get $3x+17=4x+1$

\bigskip

Now $-3x$ to get $17=x+1$

\bigskip

So $x=16$
\end{exmp}

\begin{exmp}
Solve the equation $4x+7=8-2(2x-5)$

\bigskip

First expand the brackets to get $4x+7=8-4x+10$

\bigskip

Now simplify both sides of to get $4x+7=18-4x$

\bigskip

Now $+4x$ to get $8x+7=18$


\bigskip

Now $-7$ to get $8x=11$

\bigskip

So $\displaystyle x=\frac{11}{8}$
\end{exmp}

\subsection{Exercise}
Find the value of $x$ in each equation:
\begin{multicols}{2}
\begin{enumerate}
	\item $3(x+4)=27$
	\item $5(x+10)=55$
	\item $5(x+7)=45$
	\item $10(x+4)=20$
	\item $7(x-4)=42$
	\item $3(x-5)=15$
	\item $2(x-6)=0$
	\item $9(x+2)=99$
	\item $5(2x-6)=10$
	\item $3(3x-5)=21$
	\item $4(2x+1)=28$
	\item $6(5x-8)=12$
	\item $7(3x+4)=28$
	\item $9(4x+2)=198$
	\item $3(5x-10)=15$
	\item $2(5+x)=14$
	\item $2(x+5)+3(2x-7)=5$
	\item $3(x+4)-3(2x-5)=21$
	\item $4(x-3)+3(2x-5)=13$
	\item $5(2x-6)+4(x+5)=32$
	\item $5(x+7)-4(x-1)=22$
	\item $5(2x-5)+x+1=20$
	\item $5(2x-6)+27=7$
	\item $7(x-5)-5(2x-10)=-9$
	\item $5(x+6)=4(2x+6)$
	\item $4(x-5)=6(2x-14)$
	\item $6(3x-3)=2(x+15)$
	\item $4(2x+4)=3(x+7)$
	\item $12(x-3)=3(2x-4)$
	\item $5(3x+1)=10(x+2)$
\end{enumerate}
\end{multicols}

\subsection{Linear equations with fractional terms}
When we introduce fractional terms the problems immediately feel more complicated, our aim here, as always, is to turn these problems into one of the earlier types.  Our first step is to multiply every term by every denominator, this will rid the equation of fractional terms.

\begin{exmp}
Solve the equation $\displaystyle \frac{x+6}{3}=x-4$

\bigskip

First we need to multiply each term by 3 to get $x+6=3x-12$

\bigskip

Now we $-x$ from both sides to get $6=2x-12$

\bigskip

So $2x=18$

\bigskip

So $x=9$
\end{exmp}

\begin{exmp}
Solve the equation $\displaystyle \frac{2x+5}{7}=\frac{x+4}{2}-3$

\bigskip

First we must multiply every term by 7 and 2 to get 

\bigskip

$2(2x+5)=7(x+4)-3\times 7\times 2$

\bigskip

Now expand and simplify  to get $4x+10=7x-14$

\bigskip

So $10=3x-14$

\bigskip

So $3x=24$

\bigskip

So $x=8$
\end{exmp}

\subsection{Exercise}
Find the value of $x$ in each equation:
\begin{enumerate}
	\item $\displaystyle \frac{x+1}{3}=4$
	\item $\displaystyle \frac{x}{5} + 3 = \frac{x-2}{2}$
	\item $\displaystyle x-4 = \frac{2x+3}{4}$
	\item $\displaystyle \frac{2x-5}{3} = \frac{x}{6}$
	\item $\displaystyle \frac{3x-1}{3} = x + 5$
	\item $\displaystyle \frac{5x+2}{4} = \frac{2x}{3}$
	\item $\displaystyle \frac{4x-3}{2} = \frac{x+1}{2}$
	\item $\displaystyle \frac{5-2x}{2} = x-5$
	\item $\displaystyle \frac{6}{x+1}=7$
	\item $\displaystyle \frac{5}{3x+2} = \frac{4}{2x+3}$
\end{enumerate}