\chapter{Number 2}
\section{Percentages}
Percent is just the joining of two words, 'per' and 'cent'.

Per means divide by.

Cent means 100.

So percent means divide by 100.

\subsection{Conversion to fractions}
If we remember our work from fractions we will know that a fraction is a division. So to change a percentage to a fraction we just put it over 100.

However, if the percentage is a decimal we will need to multiply, the numerator and denominator, by a power of 10 (10, 100, 1000, etc) so that no decimal remains.

\begin{exmp}
Change $23\%$ to a fraction.

\bigskip
\noindent So we put 23 over 100 and get $\displaystyle \frac{23}{100}$
\end{exmp}

\begin{exmp}
Change $4.7\%$ to a fraction.

\bigskip
\noindent So we put 4.7 over 100 and get $\displaystyle \frac{4.7}{100}$

\noindent We have a decimal in our answer so we will multiply the numerator and denominator by 10 and get the answer:
\bigskip
$\displaystyle \frac{47}{1000}$
\end{exmp}
\subsection{Exercise}
Change the following percentages to fractions:
\begin{multicols}{4}
\begin{enumerate}
	\item $48\%$
	\item $76\%$
	\item $5\%$
	\item $17\%$
	\item $7.8\%$
	\item $15.2\%$
	\item $0.4\%$
	\item $2.04\%$
	\item $0.05\%$
	\item $3.087\%$
	\item $256\%$
	\item $100\%$
\end{enumerate}
\end{multicols}
\subsection{Conversion to decimails}
We already know that per means divide and cent means 100, so percent means divide by 100.  It follows that to change a percentage to a decimal we just divide it by 100.

\begin{exmp}
Change 7.4\% to a decimal.

\bigskip
\noindent $7.4\% = 7.4 \div 100 = 0.074$
\end{exmp}
\subsection{Exercise}
Change the following percentages to decimals:
\begin{multicols}{4}
\begin{enumerate}
	\item $48\%$
	\item $76\%$
	\item $5\%$
	\item $17\%$
	\item $7.8\%$
	\item $15.2\%$
	\item $0.4\%$
	\item $2.04\%$
	\item $0.05\%$
	\item $3.087\%$
	\item $256\%$
	\item $100\%$
\end{enumerate}
\end{multicols}
\subsection{Percentage of an amount}
To find the percentage of an amount we need to understand what the word 'of' means mathematically.  It simply means $\times$.  So to find the percent of an amount we divide the percentage by 100 and then times by the amount.
\begin{exmp}
Find 34\% of 167kg

This becomes $34 \div 100 \times 167$kg

Answer: 56.78kg
\end{exmp}
\subsection{Exercise}
Calculate the following.  Show your working.
\begin{enumerate}
	\item find 10\% of 45g.
	\item what is 15\% of 180cm.
	\item find 7\% of 42 secs.
	\item find 73\% of \$50.
	\item what is 25\% of 80m.
	\item what is 48\% of 1350ml.
	\item find 36\% of 18.42m.
	\item a boy pays for 80\% of a \$15 toy.  How much does he pay?
	\item what is less? 15\% of 40cm or 23\% of 25cm.
	\item how much is 135\% of 486kg?
\end{enumerate}
\subsection{Percentage increase and decrease}
If at the beginning of our problem we have 100\% of our thing/stuff, then if we increase or decrease the amount we change the percentage amount we have.
So if I want to increase something by 12\%, then that means I now want 112\% of that thing.  On the other hand if I you decrease an amount 12\% we would want 88\% of that thing.

\begin{exmp}
Increase 120g by 15\%.

Because we are increasing we need to add 15 to 100.  So we need 115\%

So we need 115\% of 120g.

$115 \div 100 \times 120$g

Answer: 138g
\end{exmp}

\begin{exmp}
Decrease \$48 by 20\%.

Because we are decreasing we need to subtract 20 from 100.  so we need 80\%.

So we need 80\% of \$48.

$80 \div 100 \times 48$

Answer: \$38.40
\end{exmp}

\subsection{Exercise}
Increase the following amounts by 5\%.
\begin{enumerate}
	\item The cost of a \$17 train ticket
	\item \$57.50
	\item 150g
	\item 375cm
	\item 68L
	\item 1300 pupils
\end{enumerate}
Decrease the following amounts by 35\%
\begin{enumerate}
	\setcounter{enumi}{6}
	\item The price of a \$150 dress
	\item \$35
	\item 237ml
	\item 2854km
	\item 46$cm^2$
\end{enumerate}

\subsection{Reverse percentages}
When we perform a percentage increase or decrease we multiply our amount by a value (e.g. to increase an amount by 15\% we multiply by 1.15).  If we know the amount of something has be increased or decreased by, we therefore know what is was multiplied by, it follows then that if we want to find the original amount we simply need to divide by the value we would of multiplied by.
\begin{exmp}
	In a shop all prices have been decreased by 20\%.  You see a phone you would like to buy now priced at \$260.  What was the price before the sale?
	
	We know that to decrease the price by 20\%, we needed 80\%, so that means we multiplied by 0.80.  To find the original amount then we must do the opposite and divide the sale price by 0.80.
	
	Original price = $260 \div 0.80 = $ \$325
\end{exmp}
\subsection{Exercise}
The following prices have been decreased by 30\%.  Find the original prices
\begin{enumerate}
	\item \$50
	\item \$460
	\item \$90
	\item \$1250
	\item \$5
	\item \$105000
\end{enumerate}
The following prices have been increased by 15\%.  Find the original prices
\begin{enumerate}
	\setcounter{enumi}{6}
	\item \$50
	\item \$460
	\item \$90
	\item \$1250
	\item \$5
	\item \$105000
\end{enumerate}
\section{Ratios}
We often want to split up amounts into given proportions, this is dividing into given ratios.  An example might be, 'Share \$28 between two people into the ratio 2:5'.  This means that for every \$2 the first person gets the second gets \$5.  This is the same as saying that the first person get $\displaystyle \frac{2}{7}$ of the total and the other gets $\displaystyle \frac{5}{7}$

Steps:

\begin{enumerate}
	\item Find the total of the parts.  In the example above that was $2+5=7$
	\item Find the fraction for the first part.  That is the first part over the total.
	\item Find the fraction for the second part.  That is the second part over the total.
	\item Find out these fractions of the total by doing the calculations.  For the example above it would be $\displaystyle \frac{2}{7} \times \$28$ and $\displaystyle \frac{5}{7} \times \$28$
\end{enumerate}
\begin{exmp}
Split 20L of water into the ratio 1:2:5

The total of the parts is $1+2+5=8$

\bigskip

So we need $\displaystyle \frac{1}{8} \times 20L = 2.5lL$

\bigskip

and $\displaystyle \frac{2}{8} \times 20L = 5L$

\bigskip

and $\displaystyle \frac{5}{8} \times 20L = 12.5L$
\end{exmp}
\subsection{Exercise}
Split the following amounts into the given ratios
\begin{enumerate}
	\item \$15 in the ratio 2:3
	\item 30kg in the ratio 1:5
	\item 20km in the ratio 4:1
	\item 36cm in the ratio 2:3:4
	\item 40mm in the ratio 3:5
	\item 25g in the ratio 3:7
	\item \$50 in the ratio 3:4:5
	\item 30L in the ratio 1:2:5
\end{enumerate}
\section{Order of operations}
Let's consider the expression $3+4 \times 5$.  Now depending on whether we do the addition first or the multiplication we get two different answers, 35 and 23.  So which is the correct answer?  To help with this mathematicians have agreed rules on the correct order.  To help us remember the order we can use the word BEDMAS

\begin{itemize}
	\item Brackets
	\item Exponents
	\item Division
	\item Multiplication
	\item Addition
	\item Subtraction
\end{itemize}

When evaluating an expression we calculate the parts at the top of this list first.  So $3+4 \times 5$ becomes $3 + 20 = 23$.  However, if the expression was $(3+4) \times 5$ becomes $7 \times 5 = 35$.

Before we can look at some more difficult examples we need to talk about 'Exponents'.  Exponents are sometimes called powers or indices.  You can recognise them because they are that little number that you sometimes see after a number like $3^5$, where the 5 is the exponent.  What this means is that we multiply the 3 by itself 5 times.

So

$4^3$ means $4 \times 4 \times 4$

and $3^4$ means $3 \times 3 \times 3 \times 3$

Now back to the order of operations

\begin{exmp}
Evaluate $(1+ 4 * 5) \div 3$

\bigskip

The first thing we must evaluate is the bracket because brackets come first in the list.  Within the bracket we do the multiplication first because multiplication comes before addition.  So the expression becomes:

\bigskip

$(1 + 20) \div 3$

The first thing to evaluate is the bracket, so we get:

\bigskip

$21 \div 3$

\bigskip

Which is 7
\end{exmp}

\begin{exmp}
Evaluate $3^2 \times (5+2)$

\bigskip

First we deal with the bracket to get:

\bigskip

$3^2 \times 7$

\bigskip

Now we work out the exponent to get:

\bigskip

$9 \times 7$

\bigskip

Which is 63
\end{exmp}

\begin{exmp}
Evaluate $3 + 6 \times 4 - 5$

\bigskip

First we do the multiplication to get:

\bigskip

$6 + 24 - 5$

If we are left with just additions and subtractions we just work from left to right. So the answer is:

\bigskip

25
\end{exmp}

\begin{exmp}
Evaluate $(2 + 7) \div 3 \times 5$

First we deal with the bracket to get:

\bigskip

$9 \div 3 \times 5$

\bigskip

When we are left with only divisions and multiplications we work from left to right.  So the answer is:

\bigskip

15
\end{exmp}

\subsection{Exercise}
Evaluate the following expressions:
\begin{enumerate}
	\item $27 + 8 \div 4$
	\item $2^3 + 3^2$
	\item $22 - 3 \times 7 + 1$
	\item $3^3 - 4 \times 5$
	\item $(2+6)^2 \div 4$
	\item $(2 + 3) \times (5^2 - 3 \times 6)$
\end{enumerate}
